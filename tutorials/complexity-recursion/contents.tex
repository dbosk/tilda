\mode*

\section{Körtider}

\begin{frame}
  \begin{exercise}
    \begin{itemize}
      \item En viss algoritm tar 0.5 ms när antal indata är 100.
      \item Hur lång kommer körtiden att bli när antal indata är 500 om man vet 
        att körtiden är:
        \begin{enumerate}
          \item linjär, \(O(N)\);
          \item \(O(N \log N)\);
          \item kvadratisk, \(O(N^2)\);
          \item kubisk, \(O(N^3)\)?
        \end{enumerate}
    \end{itemize}
  \end{exercise}
\end{frame}

\begin{frame}
  \begin{exercise}[Tidstentan, nivå E]
    \begin{itemize}
      \item Antalet binära strängar av längd~\(n\) är \(2^n\).
      \item Antalet alfanumeriska (ca 32 olika tecken) strängar av längd~\(m\) 
        är \(32^m\).
    \end{itemize}
    \begin{enumerate}
      \item Vad blir uttrycken ovan om \(n=10\) och \(m=3\)?
      \item Hur mycket större än \(m\) måste \(n\) vara för att det första 
        uttrycket ska bli större än det andra?
        (Varför?)
    \end{enumerate}
  \end{exercise}
\end{frame}

\begin{frame}
  \begin{exercise}[Tidstentan, nivå C]
    \begin{itemize}
      \item Linda berättar för Alexander att hon bytt till ett lösenord med 45 
        binära siffror.
      \item \enquote{Men herregud!}, säger Alexander, \enquote{det blir ju 
          alldeles för lätt att knäcka. Själv har jag 9 alfanumeriska tecken i 
        mitt.}
    \end{itemize}
    \begin{enumerate}
      \item Vems lösenord är svårast att knäcka med totalsökning? (Varför?)
      \item Ungefär hur många sekunder tar knäckningen om datorn prövar en 
        miljard lösenord i sekunden? (Varför?)
    \end{enumerate}
  \end{exercise}
\end{frame}


\section{Rekursion}

\begin{frame}
  \begin{exercise}[GCD/SGD]
    \begin{itemize}
      \item För att beräkna största faktorn som två tal m och n har gemensamt 
        kan vi använda Euklides algoritm.
      \item Om \(m\) är jämnt delbart med \(n\) så är \(n\) den största 
        gemensamma faktorn.
        Annars är \(GCD(m, n) = GCD(n, m\mod n)\).
    \end{itemize}
    \begin{enumerate}
      \item Skriv en rekursiv funktion!
    \end{enumerate}
  \end{exercise}
\end{frame}

\begin{frame}[fragile]
  \begin{solution}
    \inputminted{python}{src/gcd.py}
  \end{solution}
\end{frame}

\begin{frame}
  \begin{exercise}[Fibonacci]
    \begin{itemize}
      \item Leonardo Fibonacci skrev år 1225 en bok där han beskrev denna 
        intressanta talföljd för kaninpars förökning.
      \item \(f(0) = 0, f(1) = 1, f(n) = f(n-1)+f(n-2)\)
    \end{itemize}
    \begin{enumerate}
      \item Skriv en rekursiv funktion för att beräkna Fibonaccital.
      \item Visa vilka rekursiva anrop den ger upphov till vid beräkningen av 
        f(5).
      \item Är det här det effektivaste sättet att beräkna Fibonaccitalen?
    \end{enumerate}
  \end{exercise}
\end{frame}

\begin{frame}[fragile]
  \begin{solution}[Rekursiv Fibonacci]
    \inputminted{python}{src/fib.py}
  \end{solution}
\end{frame}

\begin{frame}[fragile]
  \begin{solution}[Rekursiv Fibonacci med utskrift]
    \inputminted{python}{src/fib_print.py}
  \end{solution}
\end{frame}

\begin{frame}[fragile]
  \begin{solution}[Effektivare Fibonacci]
    \inputminted{python}{src/fib_efficient.py}
  \end{solution}
\end{frame}

\section{Binära sökträd}

\begin{frame}
  \begin{exercise}
    \begin{itemize}
      \item Skriv ihop (rekursiv) pseudokod\footnote{%
          Ja, ni får skriva i Python också.
        } för att sätta in ett element i ett binärsökträd.
    \end{itemize}
    \begin{enumerate}
      \item Hur ser det binärträd ut som skapas om man puttar in talen 4 2 1 6 
        3 7 5 i denna ordning?
      \item Och hur ser det ut om man sätter in dom i omvänd ordning, alltså 5 
        7 3 6 1 2 4?
      \item Är de balanserade?
    \end{enumerate}
  \end{exercise}
\end{frame}

\begin{frame}
  \begin{solution}
    \inputminted[firstline=22,lastline=34]{python}{src/bst.py}
  \end{solution}
\end{frame}

\begin{frame}
  \begin{exercise}
    \begin{itemize}
      \item Skriv en \emph{rekursiv} funktion som
        \begin{enumerate}
          \item skriver ut ett träd i inordning.
          \item skriver ut ett träd i preordning.
          \item skriver ut ett träd i postordning.
        \end{enumerate}
    \end{itemize}
  \end{exercise}
\end{frame}

\begin{frame}[fragile]
  \begin{solution}
    \inputminted[firstline=45,lastline=61]{python}{src/bst.py}
  \end{solution}
\end{frame}

